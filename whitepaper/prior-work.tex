\section {Related/Prior Work}

%\subsection {Concurrent Work}
Prior text mining on LDS religious documents include \cite{hilton-2008-intertext-abinadi}, which aimed to identify intertextual similarity (plagiarism, paraphrase, etc.) between chapters of LDS-specific texts. In his work, Hilton focused on \textit{The Book of Mormon}, although he has since demonstrated that similar results exist between \textit{The Holy Bible} and \textit{The Book of Mormon} \cite{hilton_2008_intertext_psalms}. Although topic models were not employed in this work, it probably could have benefitted from it. Notwithstanding, computational analysis was involved.

In concurrent work, I aim to use a combinational approach of gene sequence alignments and machine learning to automatically identify alignment positions of intertext. Although this is not directly related to this work, it has increased my familiarity with the LDS documents, dataset metadata, dataset format, and data warehouse--all of which have proven to be helpful in works such as this.

\subsection {Topic Analysis Over Time}
Hall et al. aptly demonstrated that topic entropy, when applied to topics on a per-year basis, could be used to describe the ebb and flow of each topic's popularity over time \cite{hall-jurafsky-manning:2008:EMNLP}.

\subsection{Topic Analysis on \textit{LDS Religious Documents}}
To the best of my knowledge nothing has been done in this area using the methodology we employ in this work. %This is a novel area of research.

\subsection{Venue Comparison (in Light of Estimated LDA Parameters)}
Again, this is where \cite{hall-jurafsky-manning:2008:EMNLP} shine. They showed the JS divergence was helpful in comparing/contrasting the distribution of topics and between datasets.
