\section{Test Items}
The words used to elicit antonyms are provied in Table~\ref{tab:key-words}.  The reason that these adjectives were chosen is that they are fairly non-ambiguous, with the one exception of \textit{hot}.  (Interestingly, 2 of the other 4 adjectives actually end up receiving more unique antonyms as seen in Table~\ref{tab:all_adjective}.) Although this ambiguity may have decreased the significance of the research, it does not invalidate it.  Furthermroe, significance will be shown later. These are also words that are likely to be familiar to the native.  The same is true for the nouns.  The adjectives are highly abstract while the nouns are concrete and visible so much so that you can probably search for pictures of each of them online and \textit{see} them.   

\begin{table}
	\begin{center}
		\begin{tabular}{|c|c|}	\hline
			\textbf{Adjectives} & \textbf{Nouns} 	\\ \hline
			hot					&  zipper 			\\ \hline
			black				&  sky				\\ \hline
			tall				&  shoe				\\ \hline
			large				&  robot			\\ \hline
			ugly				&  carpet			\\ \hline
		\end{tabular}
	\end{center}
	\caption{Prompt words used to elicit antonyms.}
	\label{tab:key-words}
\end{table}
%\item Abstract Nouns
%	\subitem deceit
%	\subitem hope
%	\subitem love
%	\subitem curiosity
	%\subitem dedication
	%\subitem trust
	%\subitem relaxation