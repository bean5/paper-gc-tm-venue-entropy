\section {Conclusion}
%\textbf{}
As seen in Figures 3-5%~\ref{fig:pencil_responses, fig:robot_responses, fig:sky_responses, fig:zipper_responses}
, there is substantial agreement within the first attempt to agree on a prototypical opposite.  There is also always a clear runner-up, even if the runner-up is N/A.  

Intuitively, one would find it difficult to say that antonyms exist for non-modified nouns.  However, it seems that in spite of the fact that they are not institutionalized, there is a general consensus that an opposite exists.  Clearly, respondents were able to accept the possibility of antonyms for non-modified nouns. 

Since the underlying rules which allow for the creation of an opposite are being applied to parts of speech to which they generally wouldn’t be applied, the abstractness of application allows for creativity.  The sheer number of categories antonym types is higher for concrete nouns than it is for adjectives.  A tak-like process is allowing for the selection/generation of antonyms.

This has great impact in the realm of semantic primitives.  The psychological reality of antonyms for concretes lends credence to the possibility that any noun can have an opposite, although the form of the opposition varies.

