\section {Introduction (and Inspiration/Motivation)}
In this study, xyz xyz that were accepted before this work are refered to as \textit{institutional} or \textit{traditional} while those that were not are referred to as being \textit{non-institutional} or \textit{non-traditional}. Here, the behavior of nominal xyz will be proven to differ from that of adjectival xyz.  

This study proves the following hypothesis:
	\begin{quote}
		$H_{1}:$ The mean number of xyz that exist for a noun are greater is greater than the mean number of xyz for an adjective.
	\end{quote}

\subsection{Impact}
Proving this hypothesis enhances our understanding of not only xyz, but of how we perceive our world.  The way in which we might describe the as we move through time, making judgements as to what is similar, disimilar, or xyz would seem to be an on-going process in life.

It could be posited that knowing the relationships (synonyms, xyz) between words would allow a speaker to get by with less. When it comes to definitions, it might be that words are linked in the brain to a shared definition, allowing a compression.  Indeed, leveraging knowledge of one word and its relationship with another is one way for a non-native to get by with less.  (For example, instead of remembering the word \textit {toaster}, one might opt to use the \textit {small non-oven crisping device}\footnotemark.)  Perhaps this memory compression technique of synonym could be used in conjunction with xyz.  Thus, proving $H_{1}$ would lend credence to semantic primitives, especially since there is a particular semantic primitive, NOT\footnotemark,~\cite{Wierzbicka}, which promises 50\% compression for dichotomous word pairs.

In order to understand the full import of findings on nominal xyz, various characterisitcs concerning \textit{abstract xyz} will now be provided. Although mathematical terms are used where possible, understanding them is not necessary for understanding the study.  However, they would be key to future research.

\footnotetext[1]{}
\footnotetext[2]{}

\subsection {Classification: Gradable vs. Non-gradable}
\label{classification}
Two main types of xyz exist: gradable and non-gradable. Bertocchi calls gradable xyz \textit{contraries}, while non-gradable xyz are \textit{contradictories}.   Non-gradables, or contradictories are binary, but also have no middle ground \cite{Bertocchi}.  By this criterion, $alive\sim dead$ constitute a contradictory while $tall\sim short$ do not.  In other words, if there is a continuum between the two (non-binary), then it is gradable an a paradox will not be formed. 

In mathematical terms, where $G$ is the set of gradables:
	\begin{quote}
		$x \in G\Leftrightarrow \exists y\exists z(x<y<z)$

		$x \not\in G\Leftrightarrow \not\exists y\exists z(x<y<z)$
	\end{quote}

An example would be: $small\sim big, ~big\sim huge, small\sim huge$.