\section {Introduction}% (and Inspiration/Motivation)}
%\textbf{}
In this study, antonym pairings that were accepted before this work are refered to as \textit{institutional} or \textit{traditional} while those that were not are referred to as being \textit{non-institutional} or \textit{non-traditional}. Here, the behavior of nominal antonymy will be proven to differ from that of adjectival antonymy.  

%Since nouns such as $male$ \opp $female$, $feminine$ \opp $masculine$, $boy$ \opp $girl$ are generally accepted to be pairs of opposites, the term non-modified noun will be used to refer to non-gender nouns or nouns that are not modified by an adjective (e.g. red carpet).  Meanwhile, the term noun will be used in the general sense to refer to any noun, including abstract nouns.  The tilde, \oppnospace, will be used to denote an antonym pairing such as $hot$ \opp $cold$.  

In order to understand the full import of findings on nominal antonym, various observations will now be provided concerning \textit{abstract antonymy}. 

Although mathematical terms are used where possible, understanding them is not necessary for understanding the study.  They are meant to be observational and future work will be required to determine whether the same concepts apply to nominal antonymy.

This study proves the following hypothesis:
	\begin{quote}
		$H_{1}:$ The mean number of antonyms that exist for a noun are greater is greater than the mean number of antonyms for an adjective.
	\end{quote}

\subsection{Impact}
Proving this hypothesis enhances our understanding of the way that we perceive our world and the way in which we might describe it as we move through time.  It could be posited that knowing the relation of things would allow a speaker to get by with less. This is a type of compression that would probably be helpful to memory as well as second-language acquisition.  (For example, instead of remembering the word \textit {toaster}, one might opt for the \textit {small non-oven crisping device}\footnotemark.)  As an aside, this lends credence to semantic primitives \cite{Wierzbicka} since NOT\footnotemark~is a semantic primitive---a semantic unit which could allow a compression of up to 50\%.  Learning to get get by with less might occur during second language acquisition and would undoubtedly rely on the fact that antonyms exist for both concrete and abstracts.  

\footnotetext[1]{It would also interesting to classify the types of antonyms that exist for abstract nouns; however, this is left to future work.}  %Furthermore, understanding which features are most salient about an item seem to be a good indicator as to how to best go about deriving an opposite (e.g. for \textit{big house} the best opposite is probably \textit{small house} since height is so salient; for sky the salient feature is elevation, hence both the ocean or ground are seen as viable antonyms)}
\footnotetext[2]{Go ahead and ask people to name what you are describing and see if they come up with \textit{toaster}}
\footnotetext[3]{As an aside, the pairs that Wierzbicka names such as GOOD \opp BAD are redundant if NOT is thought of an a way to oppose an item.  Wierzbicka states that a semantic primitive can only be defined by itself.  Indeed, with NOT as a semantic primitive, BAD can be simply defined as NOT GOOD (the opposite of GOOD) so it is not a semantic primitive.  However, one might do the opposite with GOOD, defining it as NOT BAD.  Determining which of the two is the semantic primitive is outside the scope of this study and perhaps not as important as proving whether semantic primitives exist (since each are equally efficient as proposed semantic primitive)}


%Sections~\ref{usages}-\ref{middle-ground} lend themselves to being described in mathematical terms while sections~\ref{usages}-\ref{middle-ground} do not.

%will follow concerning the possibility that abstracted application of the rules that govern institutionalized opposites is what govern the majority (if not all) of antonyms for unmodified nouns.  In other words, this study takes the step to determine whether rules for generation of standard antonyms can be applied more generally to even the most concrete of nouns, rules which must act in tak-like manner in order to apply more generally than they normally would.

\subsection {Selected Antonym Usages: Paradox and Middle-Ground} 
\label{usages}
Two usages of antonyms will be touched on here, \textit{paradox} and \textit{middle-ground}. For a discussion on other usages such as implicit and explicit comparison, \citeA<see>{Zhang}.  

Antonyms can be used together to create two phenomena: (1) a paradox, and (2) a middle-state or middle-ground.  It is up to natives of a language to determine which usage is created by a given pair.  Gradability probably plays a role in this.  This will be explored in section~\ref{classification}.

\subsubsection{Paradox}
Citing a Latin example \textit{neque vivus neque mortuus} (not alive not dead), Bertocchi notes that antonyms can be used to create a paradox \cite{Bertocchi}.  

\subsubsection{Middle-Ground}
The same construction that can be considered paradoxical can also be used to suggest a middle-ground.  Take for example the case of $hot$ \opp $cold$, forming the phrase \textit{not hot not cold}.  This is simply another way to express the middle-ground, \textit{warm}.  Interestingly, a word for the middle-state need not exist to detain the pair from creating a paradox.  Take the pair $tall$ \opp $short$.  One might say that the obvious between-state is \textit{medium}.  However, that does not always work for although one can say ``I am tall/short'', one cannot simply say ``I am medium'' without the accompaniment of the word \textit{height}.  Whether this is an exception or phenomenon in English is left to future work.  Nevertheless, the fact stands that in English an exact word for the middle-state need not exist.  This hints that a given antonym pair can create either a paradox or a middle-ground and it is up to the native to determine which it is.  

%\subsubsection{Discussion: Creating Middle-Ground for Nouns}
%Preliminary work showed that antonyms for nouns exist (not published; xyz).  Let's try to create a middle-ground for $pencil$ \opp $pen$.  Using the aforementioned syntax construction, we would end up with \textit{neither pencil nor pen}.  Indeed, this seems to work superficially, but one might ask whether one could use a paraphrase to describe what is really meant (is it an erasable pen or a permanent pencil?).  
%The speaker does not choose to select something which would fit the criteria, perhaps because they do not know that erasable pens exist.  The speaker uses opposition, two extremes, to create an enclosure automatically defining the possible characteristics (or non-characteristics) that are inherent in the desire.  In this case, the speaker highlights key characteristics by saying ``something that looks like a pen, but that is still erasable.''

\subsection {Classification: Gradable vs. Non-gradable}
\label{classification}
Two main types of antonyms exist: gradable and non-gradable. Bertocchi calls gradable antonyms \textit{contraries}, while non-gradable antonyms are \textit{contradictories}.   Non-gradables, or contradictories are binary, but also have no middle ground \cite{Bertocchi}.  By this criterion, $alive$ \opp $dead$ constitute a contradictory while $tall$ \opp $short$ do not.  In other words, if there is a continuum between the two (non-binary), then it is gradable an a paradox will not be formed. 

In mathematical terms: 
	\begin{quote}
		The set of \textit{gradables}, $G$, is defined $s.t.~ \forall x \exists y~s.t.~and~x \in G~(x+y)/2$ exists.

		The set of \textit{non-gradables}, $G$, is defined $s.t.~ \forall x \exists y~s.t.~x\not=y, x \in G$, and $(x+y)/2$ does not exist.
	\end{quote}

%This study will reveal the existence of a greater number of categories for concrete antonyms than for abstract nouns. 

\subsection {Attributes of Traditional Antonymy} 
Besides the ability to be classified as either gradable or non-gradable, antonyms tend to binary, communicative, and transitive. These will now be defined and discussed, with examples.  

\subsubsection {Dichotomy} 
Traditional opposites tend to be binary or dichotomous.  Lindner states ``The opposition relation is built squarely on a dichotomy or binary contrast of some sort'' \cite{UpDown}. In such cases, one term can be understood by using the negated form of its opposite.  This works only when considering optimal pairing partners for a given adjective.  Take for example \textit{small}.  Both \textit{big} and \textit{large} could be accepted as opposites, but depending on the circumstance, \textit{big} or \textit{large} is the optimal partner.

In mathematical terms: 
	\begin{quote}
		$X$ \opp $Y \Rightarrow X \perp Y~s.t. \not\exists Z$ \opp $X$.
	\end{quote}

\subsubsection {Communicativity} 
To be communicative, given one adjective $X$ whose best opposite is $Y$, $Y$’s best opposite must be $X$.  This is easily the case for traditional antonyms as they tend to be contained in obviously binary sets. In mathematical notation, where  denotes two-way implication.  In mathematical terms this would be: 
	\begin{quote}
		$(X$ \opp $Y) \Leftrightarrow (Y$ \opp $X)$
	\end{quote}

\subsubsection {Transitivity} 
For adjectives (and abstract nouns), one can find the antonym for word $X$ by finding an antonym, $Y$, for one of $X$’s synonyms, $Z$.  This is a transitive process.  In a mathematical notation where ``='' denotes synonymy, this would be: 
	\begin{quote}
		$X = Z, (Z$ \opp $Y) \Leftrightarrow (X$ \opp $Y)$
	\end{quote}

%\input{introduction-institutional-paradox}

\subsubsection {Opposition in Semantic Primitives} 
Semantic primitives make up the core meaning of language \cite{Wierzbicka}.  Wierzbicka says that semantic primitives are only definable in terms of themselves.  Interestingly, in her list of semantic primitives she includes a negating mechanism, NOT \cite{Wierzbicka}.  It is this very mechanism which leads one to believe that opposition exists at the root of the meaning of language.  However, it remains to be proven how various parts of speech behave in the atmosphere of antonymy.  This is why proving the hypothesis $H_{1}$ is important.

%\section {Side A.} 

\section {Side B.} 

