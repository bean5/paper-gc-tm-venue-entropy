\section {Methodology}
%\textbf{}
An online survey was created consisting of 3 phases: demographic, free-form antonym decisions (10 possible), and feedback.  The order of non-demographic/non-feedback questions was randomized.  %A preliminary survey was conducted to discover whether the survey was too long.

\subsection{Survey Distribution} The survey was distributed on the author’s Facebook account to his friends.  It was then reposted on Facebook by two volunteers, 2 females, thus bringing it to a wider audience.  

\subsection{Survey Format and Question Types} Demographic questions requested gender, approximate age, and whether respondent’s native language was English (Yes/No question).  Only responses from native speakers of English will be analyzed here. 

The survey consisted of two question types in separate sections. In the first section, respondents were presented with a keyword. Then they were to give two antonyms in a free-form style. The respondent’s task was to provide the best opposite possible.  They were instructed to provide institutionalized opposites if they themselves believed them to be the best opposites possible. In the case that they could not come up with a opposite, they were instructed to leave it blank, write “none”, or write “N/A” (meaning “not applicable”) in which case their response was not included in the counts since a no response is not a response.  
%This happened surprisingly few times: 22 times out of a total of 360 responses. 

\subsection{Survey Smoothing \& Normalization} It was not uncommon to receive responses that differed only in terms of capitalization, plurality, or part of speech (e.g. chair, Chairs; alive, living).  These were counted to be semantically the same.  The motivation for this is that some forms of abstracts cannoth be pluralized and that antonyms are \'supposed\' to be of teh same part of speech.

Although merging decreases the total count of opposites, this mainly impacted the number of opposites for concretes which makes the results in this survey more conservative than they might otherwise be.  Thus word groups such as alive and living are referred to collectively as alive/living. 

\subsection{Content} Following are the abstract that are used in the survey:
	\begin {enumerate}
		\item Adjectives
			\subitem hot
			\subitem black
			\subitem tall
			\subitem large
			\subitem ugly
		\item Abstract Nouns
			\subitem deceit
			\subitem hope
			\subitem love
			\subitem curiosity
			%\subitem dedication
			%\subitem trust
			%\subitem relaxation
		\item Concrete Nouns
			\subitem zipper
			\subitem sky
			\subitem shoe
			\subitem robot
			\subitem carpet
	\end {enumerate}