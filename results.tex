\section {Results}
%\textbf{}
\subsection {Analysis.} Since gender differences were only apparent in introspection feedback questions, results will be analyzed as a whole for each question posed to respondents.

\subsection {Age Distribution.} Table 5.1 shows the age distribution of the respondents.  Respondents are only considered in the analysis if they have answered at least 1 non-demographic non-feedback question.

Since the majority of respondents were of ages 18-32, this analysis will focus on them.  This provides a single generational view of responses. 

\subsection {Gender Distribution.} The gender distribution for this age range is fairly even.  This selection of ages accounts for \opp86\% (72/84) of all respondents, so it is a decently-large set of data points. Table 5.2 shows the gender distribution for this age range.


	Figure 5.2. Gender distribution of respondents of ages 18-32 who speak English natively. (Qualtrics-generated)

\subsection {Charted Responses.} As described in methodology, respondents were asked to provide their first and second best opposite for each keyword.  The most common opposites from respondents will be shown in graphs.  Based on these graphs, a most salient and second-most salient will be referred to as best opposite and second best (opposite) for the keyword.  Words that are not mentioned more than once as a best opposite or as a second best opposite are grouped into the Other category.  
 
Note that in some cases N/A is a salient second response; nevertheless, it is never a salient first best response. 


	Figure 5.3. Pen is the best antonym for pencil. Eraser is second best.

According to respondents, pen is a best antonym for pencil.  Eraser is apparently the second best antonym.  If selection of antonyms is based on collocation, then these two antonyms for pencil are interesting since in 5.4 a collocate table shows that both these antonyms are in the top 10 collocates for pencil.  The pair pencil \opp pen contrast in terms of manner of functionality; they are both modern writing instruments.  $Pencil$ \opp $eraser$ contrast in a different way, though.  Their intended uses are opposed.


	Figure 5.4. Human is the best antonym for robot.  N/A is second best.

Robot is an interesting case.  Human is overwhelmingly the best opposite for robot.  However, enough respondents chose to write N/A that it overcomes even animal vying for the role of second best opposite.  This seems to act in a very binary fashion—one or the other, but nothing else.  

Future work could contrast such results as these with results for male \opp female to determine whether robot is a new intra-modified noun polarized for binary antonymization.


	Figure 5.5. Ground is the best opposite for sky, with earth as runner-up.

For sky, ground is the best opposite according to respondents.  Ground receives a significant number of second best votes as well, but since it is already the best, earth receives the status of being second best.  Sky \opp Ground are opposed spatially as are sky \opp earth.  

Some of the categories could have been combined, but were not. Ground could be merged with dirt, ocean with sea.  Even if these were to be merged, Ground and earth would be the most salient responses.


	Figure 5.6. Student is the best antonym for teacher.  N/A is second best.

Overwhelmingly, student is the best opposite for teacher.  Like robot, this seems to be a binary pairing since the N/A category is salient compared to all other options.



	Figure 5.7. Button is the best antonym for zipper. Velcro is second best.

According to respondents, button is the best antonym for zipper.  The second best antonym is pairing is zipper \opp Velcro.  Unlike pencil, this set of antonyms fit into one group: contrast in terms of manner of functionality.  

\subsection {Collocates as Force Behind Selection of Antonyms.} Since it is possible that selection antomyms for non-modified nouns are based on collocates, Tables 5.1-5 are provided.  They show the top 10 collocates for the each key word as returned by the Corpus of Contemporary American English (COCA).  Singular and plural forms that appear in the top ten are merged.  

The parametrization for COCA searches were as follows:
Words(s): 	[at*] […].[nn]
Collocates: 	[nn*] 8 left, 8 right
Sections: 	SPOKEN

This search finds noun collocates within 8 words left or right of the provided noun (placed where the ellipsis are) in COCA’s corpus of transcribed speech.  This search specifies that [at*], any article, be adjacent to the provided noun.  This limits results to collocates for nouns that are not modified by an adjective.  So if the word were carpet, the search string would be “[at*] [carpet].[nn]” and collocates for red carpet would not be returned, but collocates for the/a/an carpet would be returned.  COCA is restricted to only search through transcribed speech because the SPOKEN section is selected.

A window of 8 left and 8 right raises the plausibility that a word is a collocate of itself (a word might be mentioned in adjacent sentences).  “[at*]” requires the search results to only find collocates for nouns that have an article adjacent to them.  

 
	Table 5.1. Pen and eraser appear in top opposites and in survey responses and in COCA collocates for pencil.

	Table 5.2. Top 10 COCA collocates for robot.
	 
	Table 5.3. Top 10 COCA collocates for sky.

	Table 5.4. Student(s) appear in top opposites and in survey responses and in COCA collocates for teacher. 

	Table 5.5. Button appears in top opposites and in survey responses and in COCA collocates for zipper.

If selection of best antonym is based on collocation, then there must be some additional rule to account for the fact that the most common collocates are not selected.  In the case of pencil, the most common collocate is paper, and yet that is not the best opposite.  In the case of teacher, school(s) is the most common collocate, but it is not the best opposite.  For sky, the most common collocate is limit and planes, but those are not salient in responses from respondent.  

\subsection {Categorizing Results.} Using even this small set of antonym pairings, one can categorize their type.  The following list includes some words whose data is not shown above but for which salient best opposites exist.  Assuming antonymy can be a continuum, the second most salient words are also included below.
	•	spatial contrast
	•	sky \opp ground
	•	functional complementation (performing intended use in another manner) 
	•	pencil \opp pen
	•	potato \opp steak, meat
	•	dog \opp cat (as pets)
	•	dog \opp goldfish (as pets)
	•	zipper \opp button
	•	zipper \opp Velcro
	•	carpet \opp hardwood (floor)
	•	functional opposition (opposing intended use; neutralize each other’s function) 
	•	pencil \opp eraser
	•	zipper \opp unzip
	•	mutual exclusion (binary contrast sets only)
	•	carpet \opp hardwood (floor)
	•	robot \opp human
	•	teacher \opp student
	•	rhyme pairs*
	•	potato \opp tomato

This is merely a rough list.  Future work might reveal a more complex hierarchy, just as a hierarchy exists for tak.  

The number of categories for these words is larger than that which exists for institutionalized words.  This lends credence to the belief that the process of creating opposites is being applied more abstractly to these words than to the traditionally accepted words. Rather than only having gradable and non-gradable antonyms alone, there seems to be a preponderance of antonym types.  Furthermore, this lends credence to the possibility that there is a tak-like process that allows for antonyms at the level of non-modified nouns to apply more liberally in spite of being less institutionalized.

\subsection {Introspection Responses.} Although no major gender differences in answers were detected by this study, the survey asked respondents to answer an introspective question.  The question was “Has your view of opposite changed because of this survey?”  Figures 5.8 and 5.9 show responses (by gender) to the question.  

Figure 5.8. Answer distribution for females to the question, “Has your view of opposites changes because of this survey” (qualtrics-generated)

  Figure 5.9. Answer distribution for males to the question, “Has your view of opposites changes because of this survey” (qualtrics-generated)

In spite of the fact that gender differences are not experienced in free-form responses, women apparently believe that they are more in-tune with the way that antonymy functions in English.  This is the only gender difference discovered in this study, although it would not be surprising if there were gender differences in selection of opposites for words which most women are familiar with which most men would not be familiar.